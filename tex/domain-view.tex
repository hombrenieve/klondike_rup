\newpage{}
\section{Vista del Dominio}\label{domaintoc}

\secttoc
Vuelve al \hyperlink{toc}{Índice Principal}.

\newpage{}
\subsection{Vocabulario}
\includegraphics[width=\textwidth]{vocabulario}
Vuelve a la \nameref{domaintoc}.

\newpage{}
\subsection{Estado Inicial}
\includegraphics[width=\textwidth]{initial-state}
Vuelve a la \nameref{domaintoc}.

\newpage{}
\subsection{Estado Victoria}
\includegraphics[width=\textwidth]{final-state}
Vuelve a la \nameref{domaintoc}.

\begin{landscape}
\subsection{Juego}
\begin{center}
\makebox[\textwidth]{\includegraphics[width=\paperwidth]{game-activity}}
\end{center}
\begin{itemize}
    \item {[1]} El número de la carta en la cima del trabajo debe ser uno más que la cima de la pila a mover. El color del palo de la cima del trabajo debe ser distinto al de la cima de la pila a mover.
    \item {[2]} El número de la carta en la cima del objetivo debe ser uno menos que la carta a mover. El palo del objetivo debe ser el mismo que el de la carta a mover.
\end{itemize}
Vuelve a la \nameref{domaintoc}.
\end{landscape}

\newpage{}
\subsection{Movimiento de deck a discarded}
\includegraphics[width=\textwidth]{move-deck-discarded-sequence}
Vuelve a la \nameref{domaintoc}.

\newpage{}
\subsection{Movimiento de work a work}
% \includegraphics[width=\textwidth]{game-activity}
TODO diagrama de secuencia
Vuelve a la \nameref{domaintoc}.

\newpage{}
\subsection{Movimiento de work a target}
% \includegraphics[width=\textwidth]{game-activity}
TODO diagrama de secuencia
Vuelve a la \nameref{domaintoc}.
